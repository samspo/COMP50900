% Created 2020-10-04 Sun 18:08
% Intended LaTeX compiler: pdflatex
\documentclass[oneside]{kaobook}
\usepackage{styles/mdftheorems}
\usepackage{styles/kaorefs}
	      \mdfdefinestyle{kaowarnstyle}{
	skipabove=1.5	opskip,
	skipbelow=.5	opskip,
	rightmargin=0pt,
	leftmargin=0pt,
	%innertopmargin=3pt,
	%innerbottommargin=3pt,
	innerrightmargin=7pt,
	innerleftmargin=7pt,
	topline=false,
	bottomline=false,
	rightline=false,
	leftline=false,
	%linewidth=1pt,
	%roundcorner=0pt,
	%font={},
	%frametitlefont={},
	frametitlerule=true,
	linecolor=black,
	%backgroundcolor=red,
	fontcolor=black,
	%frametitlebackgroundcolor=red,
}
\newmdenv[
	style=kaowarnstyle,
	backgroundcolor=red!25!White,
	frametitlebackgroundcolor=red!25!White,
]{kaowarn}
\usepackage[utf8]{inputenc}
\usepackage[T1]{fontenc}
\usepackage{graphicx}
\usepackage{grffile}
\usepackage{longtable}
\usepackage{wrapfig}
\usepackage{rotating}
\usepackage[normalem]{ulem}
\usepackage{amsmath}
\usepackage{textcomp}
\usepackage{amssymb}
\usepackage{capt-of}
\usepackage{hyperref}
\usepackage{tabularx}
\author{Samuel Sponem - HEC Montréal}
\date{Mise à jour 2020-10-04}
\title{Prévoir, décider et piloter à l'aide des informations comptables - COMP50900\\\medskip
\large COMP50900 - Comptabilité financière et comptabilité de gestion}
\hypersetup{
 pdfauthor={Samuel Sponem - HEC Montréal},
 pdftitle={Prévoir, décider et piloter à l'aide des informations comptables - COMP50900},
 pdfkeywords={},
 pdfsubject={},
 pdfcreator={Emacs 27.1 (Org mode 9.4)}, 
 pdflang={French}}
\begin{document}

\maketitle
\tableofcontents

Le programme d'approvisionnement résulte du programme de production. Ce programme d'approvisionnement définit les quantités qui devront être commandées et les dates de commande. Plusieurs paramètres déterminent les choix réalisés en la matière :
\begin{itemize}
\item le coût de passation de commande (coût de lancement et coûts de gestion des commandes : personnels affectés à la gestion des commandes, à la réception, au magasinage, à la manutention, etc.). : plus ce coût est élevé et plus il semble intéressant de faire des commandes en grande quantité pour limiter le nombre de commandes réalisées ;
\item les coûts de stockage (coût de financement du stock, primes d'assurance, coûts des moyens de stockage - coût des entrepôts, du matériel de manutention -, coût de la dépréciation du stock, etc.) : plus ce coût est élevé et moins le stock doit être élevé, ce qui signifie qu'il est nécessaire de passer de nombreuses petites commandes ;
\item le coût de la rupture de stock : plus ce coût est élevé et plus il faut être prudent en matière de gestion des stocks et limiter les ruptures possibles (ce qui implique d'avoir un niveau de stock élevé).
\end{itemize}

Diverses méthodes permettent alors d'optimiser ce coût d'approvisionnement\footnote{Par exemple, \href{https://fr.wikipedia.org/wiki/Formule\_de\_Wilson}{la formule de Wilson}}.

Le programme d'approvisionnement doit prendre en 4 éléments : les commandes, les livraisons, les consommations et le stock (fin de mois). Le budget des coûts d'approvisionnement comprend :
\begin{itemize}
\item les coûts d'achat, qui dépendent souvent des quantités achetées et doivent inclure les frais liés aux achats (notamment le transport) ;
\item les coûts d'approvisionnement (de passation de commande) et de stockage.
\end{itemize}
\begin{itemize}
\item Les budgets des centres de coûts discrétionnaires et des centres d'investissement
\end{itemize}

On parle de centre de coût discrétionnaire pour les centres de responsabilité dont le niveau de dépense n'a pas de rapport direct avec le niveau d'activité. Ceci concerne essentiellement l'administration (siège, direction générale, direction financière, service des ressources humaines, etc.) et la recherche et développement. 
Du fait de l'absence de lien explicite entre le montant de ces dépenses et le niveau d'activité, les budgets sont souvent définis par reconduction des budgets de l'année précédente. L'analyse de la valeur et des coûts cachés ou la méthode des budgets base zéro permettent de s'interroger sur la pertinence de cette reconduction systématique. 
Dans le cas des budgets d'investissement, le niveau optimal de dépense pourra être défini à l'aide des méthodes de choix d'investissement suivantes :
\begin{itemize}
\item Délais de récupération (Payback) : période nécessaire pour récupérer l'investissement initial
\item Valeur actualisée nette (VAN) : somme des flux monétaires découlant d’un projet d’investissement, actualisées au taux de rendement exigé
\item Taux de rendement interne (IRR) : taux d’actualisation  pour lequel la valeur actuelle du flux de trésorerie est exactement égale l'investissement initial  (VAN  = 0)
\end{itemize}

Le délais de récupération est fréquemment utilisé. Il est particulièrement simple à calculer.
\end{document}